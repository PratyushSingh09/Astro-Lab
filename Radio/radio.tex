\documentclass[12pt,a4paper]{article}
\usepackage[utf8]{inputenc}
\usepackage{graphicx}
\usepackage{float}
\usepackage{subcaption}
\usepackage[margin=1in]{geometry}
\usepackage{hyperref}  
\hypersetup{
  pdfborder = {0 0 0}
}
\usepackage[
backend=biber,
style=alphabetic,
]{biblatex}

\addbibresource{radio.bib}

\graphicspath{{./N Body/}}
\DeclareGraphicsExtensions{.pdf,.jpeg,.jpg,.png}

\usepackage{amsmath} % equations
\usepackage{fancyhdr} % nicer page header
\usepackage{listings} % For inclusion of code


\setlength{\parindent}{0pt} % no paragraph indents
\setlength{\parskip}{1em} % paragraphs separated by one line

\newcommand\experiment{Experiment Radio astronomy} %%%%% experiment name
\newcommand\groupno{Group 3+12}       %%%%% group number
\newcommand\names{Pratyush Singh,\\
                  Proshmit Dasputpa,\\
                  Dhruv Nahar}        %%%%% full names
\newcommand\expdate{25/03/2025}    %%%%% date of experiment day

\begin{document}
\begin{titlepage}
   \begin{center}
        \vspace*{3cm}
        \Huge{\experiment}
				
        \vspace{0.5cm}
        \LARGE{Lab course protocol}
				
        \vspace{3 cm}
        \Large{\groupno}
				
        \vspace{0.25cm}
        \large{\names}
				
        \vspace{2 cm}
        \Large{\expdate}
				
        \vspace{0.25 cm}
        \Large{Advanced lab course in astronomy\\
				Eberhard Karls Universit\"at T\"ubingen}
				
				\vspace{0.1 cm}
        \Large{WiSe 2024/25}
				
       \vfill
    \end{center}
\end{titlepage}

\pagestyle{fancy}
\fancyhf{}
\setlength{\headheight}{14.5pt}
\lhead{\groupno; \experiment}

% \section*{Abstract}
% This is optional, but never longer than half a page.

\tableofcontents
\newpage

\setcounter{page}{1}
\pagestyle{fancy}
\fancyhf{}
\rhead{\thepage}
\lhead{\groupno; \experiment}

\section{Introduction}\label{sec:intro} % labels allow references, particularly important for figures and tables


\section{Theory}
\label{sec:theory}
  \subsection{The Sun}
  \subsection{The Milky Way}
    \subsubsection{Doppler Effect}
    \subsubsection{Milkay Way Rotation}
    \subsubsection{Mass Estimate}
    % MORE IF NEEDED

\section{Experiment}
    \subsection{Observation}
        \subsubsection{The Telescope}
        \subsubsection{Observation of the Sun}
        \subsubsection{Observation of the Milky Way Disk}

    \subsection{Data Analysis}

\section{Conclusions}

\printbibliography     
\appendix
\section{Appendix}

\end{document}

